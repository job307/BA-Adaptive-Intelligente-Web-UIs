\chapter{Grundlagen}

\section{Adaptive Web UIs}

\subsection{Adaptivität vs Adaptierbarkeit}
Adaptierbarkeit
\begin{itemize}
    \item Der Benutzer nimmt Anpassungen selbst vor
    \item Der Benutzer muss sich in den Konfigurationseinstellungen auskennen
    \item Der Benutzer muss sich trauen, die Standard-Konfiguration zu verändern
    \item Meist bleiben Einstellungen erhalten, dies erfordert manuelles Zurücksetzen
\end{itemize}
Bsp.: Windows Systemeinstellungen, die Bedienungshilfen des iOS Smartphone Betriebssystems

Adaptivität
\begin{itemize}
    \item Das System verändert die Einstellungen
    \item Der Benutzer wird eventuell gefragt
    \item Der Benutzer muss sich nicht auskennen
    \item Einstellungen reagieren auf Daten
    \item Das System lernt mit
\end{itemize}
Bsp.: Wortvorschläge der iOS Betriebssystem-Tastatur, iOS Betriebssystem: Routenvorschlag
aufgrund eines Bewegungsmusters, Bennenung eines Aufenhaltspunkts aufgrund von Annahmen, Google Maps aud iOS: Anpassung des
Farbschemes je nach Tageszeit

%\section{Intelligente Web UIs}

\section{State Machines / Automaten}

\section{Statecharts / Zustandsübergangsdiagramme}

\section{Warum werden diese verwendet?}